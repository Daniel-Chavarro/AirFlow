\documentclass[conference]{IEEEtran}
\IEEEoverridecommandlockouts
% The preceding line is only needed to identify funding in the first footnote. If that is unneeded, please comment it out.
\usepackage{cite}
\usepackage{amsmath,amssymb,amsfonts}
\usepackage{algorithmic}
\usepackage{graphicx}
\usepackage{textcomp}
\usepackage{xcolor}
\usepackage{array}
\usepackage{booktabs}
\usepackage{hyperref}
\usepackage{float}
\usepackage{url}
\def\BibTeX{{\rm B\kern-.05em{\sc i\kern-.025em b}\kern-.08em
T\kern-.1667em\lower.7ex\hbox{E}\kern-.125emX}}

\hypersetup{
    colorlinks=true,
    linkcolor=blue,
    filecolor=magenta,
    urlcolor=cyan,
    pdftitle={Overleaf Example},
    pdfpagemode=FullScreen,
}

\begin{document}

    \title{Technical Documentation and System Architecture for Airflow: A Flight Reservation Management Application}

    \author{\IEEEauthorblockN{Leonardo Acevedo Monroy \\
    Daniel Alonso Chavarro Chipatecua \\
    Andrés Felipe Gómez Durán \\
    Oscar Daniel Aguas Quiroz}
    \IEEEauthorblockA{\textit{Careers of System Engineering and Computer Science} \\
    \textit{Universidad Nacional de Colombia}\\
    Bogota, Colombia}
    }

    \maketitle

    \begin{abstract}
        This paper presents the comprehensive technical documentation of Airflow, a flight reservation management system. The system is designed to efficiently manage flight reservations, providing a robust platform that enables users to search, book, and manage their air travel. The application encompasses comprehensive flight management, from aircraft and seat administration to reservation and flight status control, providinga complete experience for both end users and system administrators. This document details the database design methodology, entity-relationship modeling, system architecture, and implementation patterns used in the development process.
    \end{abstract}

    \begin{IEEEkeywords}
        flight reservation system, database design, entity-relationship model, software architecture, data access objects, ontology-based development \end{IEEEkeywords}

    \section{Introduction}
    Modern air travel management requires sophisticated software systems capable of handling complex operations, including flight scheduling, seat management, user authentication, and reservation processing. This paper presents the technical documentation for Airflow, a comprehensive flight reservation management application designed to address these requirements through a well-structured, scalable architecture.

    The system integrates multiple components including user management, aircraft inventory, flight scheduling, and reservation processing. The application serves both end users seeking to book flights and administrators requiring comprehensive system management capabilities.

    \section{Project Repository}
    The code of the entire project is in this Github repository:\\
    \url{https://github.com/Daniel-Chavarro/AirFlow}

    \section{System Architecture}

    \subsection{General Description}
    The flight reservation application architecture is based on a layered structure that clearly separates the different responsibilities of the system. The main architectural layers are as follows:

    \begin{itemize}
        \item \textbf{Presentation Layer}: User interface and interaction
        \item \textbf{Model Layer}: Data representation in the application
        \item \textbf{Persistence Layer}: Data access and management in the database
        \item \textbf{Service Layer}: Business logic
        \item \textbf{Utility Layer}: Tools for different layers \end{itemize}

    This layered architecture promotes the separation of concerns, maintainability, and scalability while ensuring clear boundaries between different system components.

    \section{Data Persistence - Database Design}

    \subsection{Development Methodology}
    The development of the entity-relationship model is based on ontologies as a conceptual foundation. This methodology enables the creation of a more coherent and semantically rich database design, ensuring that the relationships between entities accurately reflect the logic of the business domain.

    The development process followed these stages:
    \begin{enumerate}
        \item \textbf{Entity Declaration}: Identification and definition of the system's main entities
        \item \textbf{Attribute Definition}: Specification of each entity's properties with their respective data types
        \item \textbf{Relationship Establishment}: Development of connections between entities through cardinality analysis
        \item \textbf{Model Validation}: Verification of the integrity and consistency of the design \end{enumerate}

    \subsection{System Entities}
    The system comprises the following main entities:

    \subsubsection{Users}
    Represents both administrators and regular customers of the system. This entity centralizes user management with different privilege levels.

    \textbf{Attributes:}
    \begin{itemize}
        \item \texttt{id\_PK}: Unique user identifier (int, primary key)
        \item \texttt{name}: User's first name (varchar(40), not null)
        \item \texttt{last\_name}: User's last name (varchar(40), not null)
        \item \texttt{email}: Email address (varchar(200), not null)
        \item \texttt{password}: Encrypted password (varchar(20), not null)
        \item \texttt{isSuperUser}: Administrative privileges indicator (boolean, not null)
        \item \texttt{created\_at}: Account creation timestamp (datetime, not null)
    \end{itemize}

    \subsubsection{Airplanes}
    Manages information about aircraft available in the fleet.

    \textbf{Attributes:}
    \begin{itemize}
        \item \texttt{id\_PK}: Unique aircraft identifier (int, primary key)
        \item \texttt{airline}: Airline owner (varchar(20), not null)
        \item \texttt{model}: Aircraft model (varchar(50), not null)
        \item \texttt{code}: Aircraft identification code (varchar(10), not null)
        \item \texttt{capacity}: Total passenger capacity (int, not null)
        \item \texttt{year}: Manufacturing year (year)
    \end{itemize}

    \subsubsection{Cities}
    Represents cities that are part of the flight network, used to implement route graph logic.

    \textbf{Attributes:}
    \begin{itemize}
        \item \texttt{id\_PK}: Unique city identifier (int, primary key)
        \item \texttt{name}: City name (varchar(100), not null)
        \item \texttt{country}: Country where the city is located (varchar(100), not null)
        \item \texttt{code}: Airport code of the city (varchar(10), not null)
    \end{itemize}

    \subsubsection{Flights}
    Manages information about scheduled flights in the system.

    \textbf{Attributes:}
    \begin{itemize}
        \item \texttt{id\_PK}: Unique flight identifier (int, primary key)
        \item \texttt{airplane\_FK}: Reference to assigned aircraft (int, foreign key)
        \item \texttt{status\_FK}: Current flight status (int, foreign key)
        \item \texttt{origin\_city\_FK}: Origin city (int, foreign key)
        \item \texttt{destination\_city\_FK}: Destination city (int, foreign key)
        \item \texttt{code}: Flight identification code (varchar(10), not null)
        \item \texttt{departure\_time}: Departure date and time (timestamp, not null)
        \item \texttt{scheduled\_arrival\_time): Scheduled arrival date and time (timestamp NOT NULL)}
        \item \texttt{arrival\_time}: Arrival date and time (timestamp, null)
        \item \texttt{price\_base}: Base flight price (decimal(10,2), not null)
    \end{itemize}

    \subsubsection{Reservations}
    Manages reservations made by users.

    \textbf{Attributes:}
    \begin{itemize}
        \item \texttt{id\_PK}: Unique reservation identifier (int, primary key)
        \item \texttt{user\_FK}: User who made the reservation (int, foreign key)
        \item \texttt{status\_FK}: Current reservation status (int, foreign key)
        \item \texttt{flight\_FK}: Reserved flight (int, foreign key)
        \item \texttt{reserved\_at}: Reservation timestamp (timestamp, not null)
    \end{itemize}

    \subsubsection{Seats}
    Manages the available seats on the aircraft.

    \textbf{Attributes:}
    \begin{itemize}
        \item \texttt{id\_PK}: Unique seat identifier (int, primary key)
        \item \texttt{airplane\_FK}: Aircraft to which it belongs (int, foreign key)
        \item \texttt{reservation\_FK}: Associated reservation, if exists (int, foreign key, nullable)
        \item \texttt{seat\_number}: Seat number (varchar(10), not null)
        \item \texttt{seat\_class}: Seat class (enum: ECONOMY, BUSINESS, FIRST)
        \item \texttt{is\_window}: Indicates if it is a window seat (boolean)
    \end{itemize}

    \subsubsection{Flight\_Status}
    Catalog of possible states for flights.

    \textbf{Attributes:}
    \begin{itemize}
        \item \texttt{id\_PK}: Unique status identifier (int, primary key)
        \item \texttt{name}: Status name (varchar(15), not null)
        \item \texttt{description}: Detailed status description (varchar(100))
    \end{itemize}

    \subsubsection{Reservations\_Status}
    Catalog of possible states for reservations.

    \textbf{Attributes:}
    \begin{itemize}
        \item \texttt{id\_PK}: Unique status identifier (int, primary key)
        \item \texttt{name}: Status name (varchar(15), not null)
        \item \texttt{description}: Detailed status description (varchar(100))
    \end{itemize}

    \subsection{Entity Relationships}
    The system establishes the following relationships between the identified entities:

    \subsubsection{User-Reservation Relationships}
    \textbf{Cardinality}: 1:M (One to Many)
    \begin{itemize}
        \item A user can make zero or multiple flight reservations
        \item Each reservation is assigned to only one user
        \item This relationship ensures reservation traceability
    \end{itemize}

    \subsubsection{Aircraft-Reservation Relationships}
    \textbf{Cardinality}: 1:M (One to Many)
    \begin{itemize}
        \item An aircraft can be assigned to multiple reservations
        \item Each reservation can have only one aircraft assigned
        \item A reservation may initially have no aircraft assigned
    \end{itemize}

    \subsubsection{Aircraft-Seat Relationships}
    \textbf{Cardinality}: 1:M (One to Many)
    \begin{itemize}
        \item An aircraft is composed of multiple seats
        \item A seat belongs exclusively to a specific aircraft
        \item This relationship defines the physical configuration of each aircraft
    \end{itemize}

    \subsubsection{Aircraft-Flight Relationships}
    \textbf{Cardinality}: 1:M (One to Many)
    \begin{itemize}
        \item An aircraft may or may not be assigned to an active flight
        \item A flight must have an aircraft assigned
        \item This relationship manages flight operations
    \end{itemize}

    \subsubsection{Reservation-Seat Relationships}
    \textbf{Cardinality}: 1:M (One to Many)
    \begin{itemize}
        \item A reservation can include one or multiple seats
        \item A seat may or may not be assigned to a reservation
        \item This relationship enables group reservations
    \end{itemize}

    \subsubsection{Reservation-Flight Relationships}
    \textbf{Cardinality}: M:1 (Many to One)
    \begin{itemize}
        \item A reservation is assigned to a specific flight
        \item A flight can have multiple registered reservations
        \item This relationship links reservations with scheduled flights
    \end{itemize}

    \subsubsection{Status Relationships}
    \textbf{Cardinality}: 1:M (One to Many)
    \begin{itemize}
        \item A reservation has a unique status at any given time
        \item A status can be assigned to multiple reservations
        \item A flight has a specific status
        \item A flight status can apply to multiple flights
    \end{itemize}

    \subsubsection{City-Flight Relationships}
    \textbf{Cardinality}: 1:M (One to Many)
    \begin{itemize}
        \item A flight must depart from an origin city
        \item A flight must land at a destination city
        \item A city can be the origin of multiple flights
        \item A city can be the destination of multiple flights
        \item This structure enables implementation of routing algorithms and graphs
    \end{itemize}

    \subsection{Design Considerations}

    \subsubsection{Referential Integrity}
    The design implements referential integrity constraints through foreign keys, ensuring data consistency and preventing references to non-existent entities.

    \subsubsection{Normalization}
    The model is in third normal form (3NF), eliminating redundancies and transitive dependencies, which optimizes storage and reduces update anomalies.

    \subsubsection{Scalability}
    The structure allows horizontal growth through implementation of appropriate indexes and partitioning of large tables such as \texttt{reservations} and \texttt{flights}.

    \subsubsection{Flexibility}
    The use of status tables (\texttt{flight\_status} and \texttt{reservations\_status}) provides flexibility to add new states without modifying the structure of main tables.

    \section{Data Access Object (DAO) Pattern}

    The system implements the DAO design pattern to abstract and encapsulate database access, clearly separating business logic from persistence logic.
    The implementation follows these principles:

    \subsection{DAOMethods Interface}
    A generic interface \texttt{DAOMethods<T>} is defined that establishes standard methods for CRUD operations:

    \begin{itemize}
        \item \texttt{ArrayList<T> getAll()}: Retrieves all objects of type T
        \item \texttt{T getById(int id)}: Retrieves a specific object by its identifier
        \item \texttt{void create(T object)}: Creates a new record in the database
        \item \texttt{void update(int id, T toUpdate)}: Updates an existing record
        \item \texttt{void delete(int id)}: Deletes a specific record
    \end{itemize}

    \subsection{Specific Implementations}
    Each domain entity has its own DAO implementation that extends the generic interface:

    \begin{itemize}
        \item \texttt{FlightDAO}: Manages CRUD operations for flights
        \item \texttt{ReservationDAO}: Manages CRUD operations for reservations
        \item \texttt{UsersDAO}: Manages CRUD operations for users
        \item \texttt{AirplaneDAO}: Manages CRUD operations for aircraft
        \item \texttt{CityDAO}: Manages CRUD operations for cities
        \item \texttt{SeatDAO}: Manages CRUD operations for seats
    \end{itemize}

    This structure ensures uniformity in data access and facilitates code maintainability by centralizing persistence logic.

    \subsection{SQL Query Methods}
    DAOs implement various SQL queries to satisfy system requirements. Below are some relevant examples:

    \subsubsection{FlightDAO Queries}
    The FlightDAO class implements complex queries including flight searches with status information and multi-criteria filtering based on origin, destination, and date parameters.

    \subsubsection{ReservationDAO Queries}
    The ReservationDAO handles user-specific reservation retrieval and seat availability verification through join operations between multiple tables.

    \subsection{Database Connection Management}
    Database connection management is centralized in the \texttt{ConnectionDB} class, following the Singleton pattern to optimize resources. The connection configuration includes Unicode support, SSL configuration, and comprehensive exception handling.

    \subsubsection{Transaction Handling}
    The system implements transaction control for critical operations such as flight reservations, ensuring data integrity even in complex operations affecting multiple tables through commit/rollback mechanisms.


    \section{User Interface Design}

    \subsection{Design Philosophy}
    The graphical user interface has been designed following principles of minimalism and simplicity, with the aim of providing an intuitive and modern user experience. The key elements that have guided the design are:

    \begin{itemize}
        \item \textbf{Simplicity}: Elimination of unnecessary elements to focus on essential functionality
        \item \textbf{Consistency}: Coherent use of visual elements and interaction patterns throughout the application
        \item \textbf{Accessibility}: Clear and intuitive interface that facilitates navigation for all users
        \item \textbf{Modern aesthetics}: Implementation of a contemporary design using the FlatLaf library
    \end{itemize}

    \subsection{Technologies Used}

    \begin{itemize}
        \item \textbf{Java Swing}: Base framework for building graphical components
        \item \textbf{FlatLaf}: External library that provides a modern and flat look to Swing components
        \item \textbf{Responsive Design}: Implementation of layouts that adapt to different screen sizes
    \end{itemize}

    \subsection{Interface Structure}
    The user interface is organized following a hierarchical component architecture:

    \begin{enumerate}
        \item \textbf{MainFrame}: Main window that contains all other components
        \item \textbf{Horizontal Menu}: Top bar with logo, title and navigation options
        \item \textbf{Content Panel}: Central area with card system to display different sections
        \item \textbf{Specific Panels}: Components dedicated to specific functionalities
    \end{enumerate}

    \subsection{Main Components}

    \subsubsection{Flight Search Panel}
    Interface that allows users to search for available flights according to criteria such as origin, destination, and dates. Includes:
    \begin{itemize}
        \item Origin and destination city selectors
        \item Travel date selector
        \item Additional filters
        \item Search button
    \end{itemize}

    \subsubsection{Flight Details Panel}
    Displays detailed information about a selected flight, including:
    \begin{itemize}
        \item Airline information and flight number
        \item Departure and arrival times
        \item Airport details
        \item Seat selection
    \end{itemize}

    \subsubsection{Seat Selection Panel (BookSeatsPanel)}
    Interactive interface that allows users to visualize and select specific seats on the chosen aircraft.

    \textbf{Visualization features:}
    \begin{itemize}
        \item Graphical representation of the aircraft
        \item Visual differentiation by service class
        \item Real-time availability indicators
        \item Legend of symbols and colors
    \end{itemize}

    \textbf{Selection functionalities:}
    \begin{itemize}
        \item Multiple seat selection
        \item Availability validation
        \item Automatic price calculation
        \item Seat recommendations
    \end{itemize}

    \subsubsection{Confirmation Panel}
    Allows the user to review and confirm their booking details before finalizing:
    \begin{itemize}
        \item Flight information summary
        \item Passenger details
        \item Payment information
        \item Confirmation or cancellation buttons
    \end{itemize}

    \subsubsection{Dialog Components}
    The application implements specific dialog components to handle user authentication:

    \begin{itemize}
        \item \textbf{Login Dialog}: Provides a secure interface for users to authenticate and access their accounts
        \item \textbf{Register Dialog}: Allows new users to create accounts by entering their personal information
    \end{itemize}

    These dialog components follow the same minimalist design principles as the main interface, maintaining visual consistency throughout the application while serving their specialized functions.

    \subsection{Visual Design Elements}

    \begin{itemize}
        \item \textbf{Color Palette}: Predominance of light tones with subtle accents to highlight important elements
        \item \textbf{Typography}: Use of sans-serif fonts to improve readability
        \item \textbf{Iconography}: Implementation of minimalist icons to represent common actions
        \item \textbf{Spacing}: Design with sufficient white space to reduce visual overload \end{itemize}

    \subsection{User Interaction and Flow}
    The navigation flow in the application follows a linear and predictable pattern.

    \begin{enumerate}
        \item Search for available flights
        \item Selection and display of flight details
        \item Selection of seats and additional options
        \item Review and confirmation of the booking
        \item Display of confirmation and booking details
    \end{enumerate}

    This sequential design guides the user through the booking process intuitively, reducing the learning curve and minimizing possible errors during the process.

    \subsection{Interface Screenshots}
    Below are screenshots of the application's main interfaces demonstrating the implemented design principles and user flows.

    \begin{figure}[H]
        \centering
        \includegraphics[width=\linewidth]{images/UI/MainPanel.png}
        \caption{Main application interface showing the flight search panel}
        \label{fig:main_interface}
    \end{figure}

    \begin{figure}[H]
        \centering
        \includegraphics[width=\linewidth]{images/UI/DetailsPanel.png}
        \caption{Flight details panel showing selected flight information}
        \label{fig:flight_details}
    \end{figure}

    \begin{figure}[H]
        \centering
        \includegraphics[width=\linewidth]{images/UI/BookSeatsPanel.png}
        \caption{Book Seats panel showing available seats}
        \label{fig:book_seats}
    \end{figure}
    \begin{figure}[H]
        \centering
        \includegraphics[width=\linewidth]{images/UI/ConfirmationPanel.png}
        \caption{Confirmation panel showing details of reservation}
        \label{fig:confirm_interface}
    \end{figure}


    These screenshots demonstrate how the design principles of simplicity, consistency, and modern aesthetics have been applied throughout the application, resulting in a clean and intuitive user interface.

    \section{GUI Implementation with Bridge and Command Patterns}
    To enhance the modularity and maintainability of the user interface, the Bridge and Command design patterns were implemented. This section details how these patterns were integrated into the GUI architecture.

    \subsection{Bridge Pattern for Decoupling}
    The Bridge pattern was used to decouple the user interface (the "View") from its controlling logic (the "Controller"). This separation allows the UI and the business logic to evolve independently.

    \begin{itemize}
        \item \textbf{Abstraction (\texttt{View} Interface)}: An interface named \texttt{View} was created to define all the operations that the user interface must support. This includes methods for displaying panels, retrieving data from forms, and setting data for display. This interface acts as the "bridge" between the UI implementation and the controller.

        \item \textbf{Concrete Implementor (\texttt{MainFrame})}: The \texttt{MainFrame} class implements the \texttt{View} interface, providing the concrete implementation of the UI. It manages a \texttt{CardLayout} to switch between different panels like \texttt{SearchFlightPanel}, \texttt{DetailsFlightPanel}, etc.
    \end{itemize}

    This structure allows the controller to interact with the UI through the \texttt{View} interface, without being coupled to the specific implementation details of \texttt{MainFrame} or jego sub-components.

    \subsection{Command Pattern for UI Actions}
    The Command pattern was used to encapsulate all the information needed to perform an action or trigger an event. In the context of the GUI, this means that each user action (e.g., clicking a button) is treated as a command.

    \begin{itemize}
        \item \textbf{Commands}: A set of string constants (e.g., \texttt{SEARCH\_FLIGHT\_CMD}, \texttt{BOOK\_SEAT\_CMD}) are defined in the \texttt{View} interface. These constants represent the different commands that can be triggered by the user.

        \item \textbf{Invokers}: UI components like \texttt{JButton} act as invokers. Each interactive component is assigned a specific action command using the \texttt{setActionCommand()} method.

        \item \textbf{Receiver and Controller}: A single \texttt{ActionListener} in the (yet to be implemented) controller will act as the receiver for all these commands. It will inspect the action command of the event and delegate the request to the appropriate handler method, effectively executing the command.
    \end{itemize}

    By using the Command pattern, the UI components are completely decoupled from the logic that is executed. The buttons don't know what happens when they are clicked; they only know to send a specific command to the listener.

    \subsection{Dynamic Flight Display}
    The flight search results are now displayed dynamically as a series of cards.

    \begin{itemize}
        \item \textbf{\texttt{FlightCardPanel}}: A new component, \texttt{FlightCardPanel}, was created to represent a single flight in the search results. This panel displays key information like the route, time, duration, and price, along with a "View Details" button.

        \item \textbf{Dynamic Population}: The \texttt{SearchFlightPanel} now contains a method, \texttt{displayFlights()}, which takes a list of flights and dynamically creates a \texttt{FlightCardPanel} for each one. These cards are added to a results panel, which is displayed to the user.
    \end{itemize}

    This approach makes the search results more interactive and visually appealing, and it further demonstrates the flexibility of the implemented architecture. The main frame delegates the responsibility of displaying the flights to the search panel, which in turn uses the specialized card panel for the rendering of each flight.


    \begin{thebibliography}{00}
        \bibitem{b1} Date, C.J., ``An Introduction to Database Systems,'' 8th ed. Boston: Addison-Wesley, 2004.
        \bibitem{b2} Fowler, M., ``Patterns of Enterprise Application Architecture,'' Boston: Addison-Wesley, 2002.
        \bibitem{b3} Silberschatz, A., Galvin, P.B., and Gagne, G., ``Database System Concepts,'' 7th ed. New York: McGraw-Hill, 2019.
    \end{thebibliography}

\end{document}