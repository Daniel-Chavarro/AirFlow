\documentclass[conference]{IEEEtran}
\IEEEoverridecommandlockouts
% The preceding line is only needed to identify funding in the first footnote. If that is unneeded, please comment it out.
\usepackage{cite}
\usepackage{amsmath,amssymb,amsfonts}
\usepackage{algorithmic}
\usepackage{graphicx}
\usepackage{textcomp}
\usepackage{xcolor}
\usepackage{array}
\usepackage{booktabs}
\usepackage{hyperref}
\usepackage{float}
\usepackage{url}
\def\BibTeX{{\rm B\kern-.05em{\sc i\kern-.025em b}\kern-.08em
T\kern-.1667em\lower.7ex\hbox{E}\kern-.125emX}}

\hypersetup{
    colorlinks=true,
    linkcolor=blue,
    filecolor=magenta,      
    urlcolor=cyan,
    pdftitle={AirFlow Flight Reservation System},
    pdfpagemode=FullScreen,
    }

\begin{document}
\title{AirFlow: Flight Reservation System\\
{\footnotesize An Airline Reservation Management System}}

\author{\IEEEauthorblockN{Adrian}
\IEEEauthorblockA{\textit{Software Engineering Department} \\
\textit{University} \\
City, Country \\
email@university.edu}
}

\maketitle

\begin{abstract}
This document presents AirFlow, a comprehensive flight reservation system developed in Java with MVC architecture. The system provides an intuitive graphical interface that allows users to search for flights, select seats, and make reservations efficiently. The application implements robust design patterns and a relational database to ensure data integrity and consistency.
\end{abstract}

\begin{IEEEkeywords}
reservation system, Java Swing, MVC architecture, database, graphical interface
\end{IEEEkeywords}

\section{Introduction}
AirFlow is an airline reservation management system that enables users to perform flight searches, select seats, and complete reservations intuitively. The system is developed using Java with Swing for the graphical interface and MySQL for data persistence.

\section{System Architecture}
The system follows the Model-View-Controller (MVC) architectural pattern, clearly separating business logic, presentation, and application flow control.

\subsection{Data Model}
The model includes the following main entities:
\begin{itemize}
    \item \textbf{User}: Customer information management
    \item \textbf{Flight}: Available flight information
    \item \textbf{Airplane}: Aircraft specifications and capacity
    \item \textbf{Seat}: Seat configuration and availability
    \item \textbf{Reservation}: Completed reservation transactions
    \item \textbf{City}: Origin and destination information
\end{itemize}

\subsection{Data Access Layer (DAO)}
The implementation of DAO patterns ensures separation between business logic and data access, providing a consistent interface for CRUD operations.

\section{Graphical User Interface}

\subsection{Main Frame (MainFrame)}
The \texttt{MainFrame} acts as the main container of the application, coordinating navigation between different panels and maintaining the global state of the user session.

\subsubsection{Main features:}
\begin{itemize}
    \item Centralized navigation management
    \item User session control
    \item Component coordination
    \item Global event handling
\end{itemize}

\subsection{Flight Search Panel (SearchFlightPanel)}
This panel provides the main functionality for users to search for available flights according to their specific criteria.

\subsubsection{Implemented functionalities:}
\begin{itemize}
    \item Origin and destination city selection
    \item Travel date configuration
    \item Advanced search filters
    \item Data input validation
    \item Integration with search services
\end{itemize}

\subsubsection{Interface components:}
\begin{itemize}
    \item ComboBox for city selection
    \item DatePicker for travel dates
    \item Additional filtering fields
    \item Search button with validation
\end{itemize}

\subsection{Flight Details Panel (DetailsFlightPanel)}
Displays detailed information of flights found in the search, allowing the user to review all specifications before proceeding with the reservation.

\subsubsection{Information displayed:}
\begin{itemize}
    \item Departure and arrival times
    \item Airline and aircraft information
    \item Flight duration
    \item Prices by service class
    \item Seat availability
\end{itemize}

\subsubsection{Available interactions:}
\begin{itemize}
    \item Preferred flight selection
    \item Option comparison
    \item Navigation to seat selection
    \item Return to search to modify criteria
\end{itemize}

\subsection{Seat Selection Panel (BookSeatsPanel)}
Interactive interface that allows users to visualize and select specific seats on the chosen aircraft.

\subsubsection{Visualization features:}
\begin{itemize}
    \item Graphical representation of the aircraft
    \item Visual differentiation by service class
    \item Real-time availability indicators
    \item Legend of symbols and colors
\end{itemize}

\subsubsection{Selection functionalities:}
\begin{itemize}
    \item Multiple seat selection
    \item Availability validation
    \item Automatic price calculation
    \item Seat recommendations
\end{itemize}

\subsection{Confirmation Panel (ConfirmPanel)}
Final panel that consolidates all reservation information and allows the user to review and confirm their selection before final processing.

\subsubsection{Consolidated information:}
\begin{itemize}
    \item Complete reservation summary
    \item Selected flight details
    \item Reserved seats
    \item Itemized total cost
    \item Passenger information
\end{itemize}

\subsubsection{Available actions:}
\begin{itemize}
    \item Final reservation confirmation
    \item Selection modifications
    \item Process cancellation
    \item Receipt generation
\end{itemize}

\section{System Dialogs}

\subsection{Authentication Dialog (LoginDialog)}
Modal interface for user authentication in the system.

\subsubsection{Security features:}
\begin{itemize}
    \item Credential validation
    \item Secure password handling
    \item Brute force attack prevention
    \item Password recovery
\end{itemize}

\subsubsection{User experience:}
\begin{itemize}
    \item Clean and accessible interface
    \item Clear error messages
    \item Remember credentials option
    \item New user registration link
\end{itemize}

\subsection{Registration Dialog (RegisterDialog)}
Allows new users to create accounts in the system with complete data validation.

\subsubsection{Registration fields:}
\begin{itemize}
    \item Basic personal information
    \item Contact details
    \item Access credentials
    \item Initial preferences
\end{itemize}

\subsubsection{Implemented validations:}
\begin{itemize}
    \item Email format verification
    \item Password strength validation
    \item Duplicate data confirmation
    \item Terms and conditions acceptance
\end{itemize}

\section{Technical Implementation}

\subsection{Technologies Used}
\begin{itemize}
    \item \textbf{Java SE}: Main development language
    \item \textbf{Swing}: Framework for graphical interface
    \item \textbf{MySQL}: Database management system
    \item \textbf{JDBC}: Database connectivity
    \item \textbf{Maven}: Dependency management and build
\end{itemize}

\subsection{Applied Design Patterns}
\begin{itemize}
    \item \textbf{MVC}: Separation of responsibilities
    \item \textbf{DAO}: Data access abstraction
    \item \textbf{Observer}: State change notification
    \item \textbf{Factory}: Complex object creation
\end{itemize}

\section{User Workflow}

The system implements an intuitive workflow that guides the user through the complete reservation process:

\begin{enumerate}
    \item \textbf{Authentication}: User login or registration
    \item \textbf{Search}: Travel criteria specification
    \item \textbf{Selection}: Flight review and choice
    \item \textbf{Reservation}: Specific seat selection
    \item \textbf{Confirmation}: Final review and processing
\end{enumerate}

\section{Conclusions}

AirFlow presents a complete solution for airline reservation management, implementing software development best practices and providing a smooth and intuitive user experience. The modular architecture facilitates system maintenance and extension, while the well-designed graphical interface ensures accessibility for users of various technical levels.

\subsection{Future Work}
\begin{itemize}
    \item Online payment implementation
    \item Integration with external airline APIs
    \item Mobile application development
    \item Automated notification system
    \item Advanced analytics and reporting
\end{itemize}

\begin{thebibliography}{00}
\bibitem{b1} Oracle Corporation, "Java Platform, Standard Edition Documentation," 2023.
\bibitem{b2} MySQL AB, "MySQL 8.0 Reference Manual," MySQL Documentation, 2023.
\bibitem{b3} Apache Software Foundation, "Apache Maven Project," 2023.
\bibitem{b4} Gamma, E., Helm, R., Johnson, R., \& Vlissides, J., "Design Patterns: Elements of Reusable Object-Oriented Software," Addison-Wesley, 1994.
\end{thebibliography}

\end{document}
